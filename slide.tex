\documentclass{beamer}
\usepackage{graphicx}
\usepackage{listings}
\usepackage{changepage}
\usepackage{amssymb}
\usepackage{amsmath}
\usepackage{xcolor}
\usetheme{CambridgeUS}

\title{Metodi di Krylov}
\author{Loredana Cascarano}
\date{Luglio 2024}

\begin{document}

% Title Page
\begin{frame}
\titlepage
\end{frame}

% Slide: Introduzione ai Metodi di Krylov
\begin{frame}
\frametitle{Introduzione ai Metodi di Krylov}
\begin{itemize}
\item I metodi di Krylov sono algoritmi iterativi per la risoluzione di grandi sistemi lineari \(Ax = b\) e problemi agli autovalori.
\item L'idea chiave è di cercare una soluzione approssimata all'interno di un sottospazio di Krylov, generato da:
\[ K_m(A, b) = \text{span}\{b, Ab, A^2b, \ldots, A^{m-1}b\} \]
\item Sono particolarmente utili quando la matrice \(A\) è di grandi dimensioni o sparsa.
\end{itemize}
\end{frame}

% Slide: Sottospazi di Krylov
\begin{frame}
\frametitle{Sottospazi di Krylov}
\begin{itemize}
\item Un sottospazio di Krylov di ordine \(m\) associato alla matrice \(A\) e al vettore \(b\) è definito come:
\[ K_m(A, b) = \text{span}\{b, Ab, A^2b, \ldots, A^{m-1}b\} \]
\item È un sottospazio di dimensione al massimo \(m\) che contiene informazioni utili sulla soluzione del sistema lineare.
\item I metodi di Krylov sfruttano questi sottospazi per trovare soluzioni approssimate efficienti.
\end{itemize}
\end{frame}

% Slide: Metodo di Arnoldi
\begin{frame}
\frametitle{Metodo di Arnoldi}
\begin{itemize}
\item L'iterazione di Arnoldi è un metodo di Krylov che costruisce una base ortonormale per il sottospazio di Krylov \(K_m(A, b)\).
\item Questa base viene utilizzata per ridurre il problema originale ad un problema di dimensioni molto minori, coinvolgendo una matrice di Hessenberg superiore.
\item La matrice di Hessenberg superiore ottenuta ha gli stessi autovalori di \(A\), facilitando l'approssimazione degli autovalori e autovettori di \(A\).
\end{itemize}
\end{frame}

% Slide: Matrice Upper-Hessenberg
\begin{frame}
\frametitle{Matrice Upper-Hessenberg}
\begin{itemize}
\item Una matrice \(H\) è detta upper-Hessenberg se ha elementi nulli sotto la prima sottodiagonale: \(h_{ij} = 0\) per \(i > j + 1\).
\item Esempio:
\[
H = \begin{pmatrix}
h_{11} & h_{12} & h_{13} & \cdots & h_{1n} \\
h_{21} & h_{22} & h_{23} & \cdots & h_{2n} \\
0 & h_{32} & h_{33} & \cdots & h_{3n} \\
\vdots & \vdots & \vdots & \ddots & \vdots \\
0 & 0 & 0 & \cdots & h_{nn}
\end{pmatrix}
\]
\item Questa struttura è cruciale nell'iterazione di Arnoldi e semplifica il calcolo degli autovalori.
\end{itemize}
\end{frame}

% Slide: Implementazione del Metodo di Arnoldi (Python)
\begin{frame}[fragile]
\frametitle{Implementazione del Metodo di Arnoldi (Python)}
\begin{lstlisting}[language=Python, basicstyle=\footnotesize, frame=single, numbers=none]
import numpy as np

def arnoldi_iteration(A, b, n):
    m = len(A)
    H = np.zeros((n+1, n))
    Q = np.zeros((m, n+1))
    #Normalizzazione del vettore iniziale
    Q[:, 0] = b / np.linalg.norm(b)
    for k in range(n):
        v = np.dot(A, Q[:, k])
        for j in range(k+1):
            H[j, k] = np.dot(Q[:, j].T, v)
            #Ortogonalizzazione di Gram-Schmidt
            v = v - H[j, k] * Q[:, j]
        H[k+1, k] = np.linalg.norm(v)
        if H[k+1, k] != 0 and k+1 < m:
            #Normalizzazione
            Q[:, k+1] = v / H[k+1, k]
    return Q, H
\end{lstlisting}
\end{frame}

% Slide: Esempio di Applicazione
\begin{frame}[fragile]
\frametitle{Esempio di Applicazione}
\begin{lstlisting}[language=Python, basicstyle=\footnotesize, frame=single, numbers=none]
A = np.array([[4, 1, 0],
              [1, 4, 1],
              [0, 1, 4]])
b = np.array([1, 0, 0])

Q, H = arnoldi_iteration(A, b, 2)

print("Q:\n", Q)
print("H:\n", H)
\end{lstlisting}
\end{frame}

% Slide: Conclusione
\begin{frame}
\frametitle{Conclusione}
\begin{itemize}
\item I metodi di Krylov sono strumenti potenti per risolvere problemi lineari di grandi dimensioni.
\item L'iterazione di Arnoldi è un metodo fondamentale per la costruzione di sottospazi di Krylov e la riduzione della dimensionalità del problema.
\item L'implementazione in Python dimostra la praticità di questi metodi nella risoluzione di problemi reali.
\end{itemize}
\end{frame}

\end{document}
